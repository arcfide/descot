\input docmac

\title{Guide to Packaging with Descot}
\author{Shu Zhou and Aaron W. Hsu}
\contact{[shuzhou,awhsu]@indiana.edu}

\maketitlepage

\abstract
The Descot packager is designed to improve the portability and visibility 
of code that is distributed
by Scheme programmers. 
Rather than define code using a specific module 
system with a specific system dependency or layout
that must be individually translated, into either other module 
systems or into Descot metadata, the Descot packager provides an easy 
to write encoding of the core vocabulary in the Descot metalanguage in 
an extensible format. Instead of writing SRDF files for each file, 
you can define a series of package files that describe your code,
and this can be used as a common base for a variety of uses. 
A library author or collector can generate a single SRDF file suitable
for uploading to a Descot repository, making it easy to publish the
information about a library.
Someone who has downloaded the library and wishes to use it in a
possibly unique or special environment can generate modules for a
specific Scheme implementation from the package files, and can use a
package manager to install or configure the package appropriately.
This Guide introduces Descot and the problems it solves in the
Introduction, and then moves on with a tutorial about using the
packager. Following that the details of how to use the system and its
features are described. User's who wish to gain a more technical,
implementator's perspective on the program should consider the woven
source code as their best resource.
\endabstract
 
\chapter{Introduction}{}%

\section{Ben and Amy}{}%
Throughout this guide, we will revisit two characters in a common
programming story. Ben is programmer who has written some library
code, and wishes to distribute it to people who would like to make use
of this very convenient library. Amy is just such an user. As a
programmer, she does not like to do everything herself, and so she is
in the perfect position to benefit from Ben's library code, if she can
get to it and use it on her system.

\section{Ben's Problem}{}%
while Ben has written this nice little library --- or maybe it is not
so little --- he has two problems. Firstly, he wants to be able to
get the word about his software. He needs an easy to put his code in a
visible place where other programmers will see his code. Secondly,
even if other programmers can see his code, they are on different
systems, and are likely using different implementations of the
programming language that he has used to write this code. [This is
Scheme, after all.] Without a better Solution, Ben is stuck writing a
number of portability layers for his code, or others are stuck having
to do this possibly non-trivial task themselves. Ben needs an easy way
to make sure that everyone can use his code with minimal effort.

\section{Amy's Problem}{}%
Amy likes Ben's library, and would like to use it, but she has to be
able to find it easily first. Even after she finds it, she doesn't
want to go through too much trouble to get it working with her
compiler or interpreter, and she doesn't want to be tied to using it
in just the same way that Ben used it. She may want to install it to a
different location or perhaps even repackage it for distribution, if
Ben allows this sort of thing. She needs an easy way to quickly and
programmatically grasp the important and salient features of Ben's
library so that she can either modify it to suit her environment,
hopefully automatically, or go elsewhere. 

\section{Using Descot to Solve the Problem}{}%
Descot could solve the problem mentioned above for Ben and Amy. The 
following is how it works: Ben should write SRDF file by hand which 
is a really complicated job and then send it to a Descot server or send
it directly to Amy. For Amy, she will download the file in Descot
server or accept the file directly sent to her. Once she has received the
SRDF file, she has to write an SRDF2module compiler that converts the
SRDF information into her system's language. 
This solution is vastly superior to the other possibility, of having
to do it all by hand without any programmatically parseable
information, but SRDF or other encodings of RDF are not the easiest to
deal with, because they deal with the full generality of RDF.
Moreover, if a bunch of user's each write their own SRDF2module
translation program, they will have duplicated a lot of effort if they
don't share the common subsystems that would inevitably arise from
these efforts. 

\section{Using the Packager to make it easier}{}%
Under this situation, the Descot Packager is provided so the problem
mentioned above will be easier to solve. Now people like Ben have
no need to write an SRDF file by hand any more, instead he should write
a Packager file instead of an SRDF file and this is significantly easier
than writing an SRDF file. Then, the packager program will generate the
SRDF for him.

When Amy gets the archive that Ben distributes (which she found after
hunting for it in a Descot repository), these same package files give
enough information to Amy to generate modules for her system.
Moreover, if her system is not supported by default in the packager,
she can easily hook into the packager and provide her own translator.
Her task will be much easier now, because all the common work that
she might have had to do before it taken care of for her.

\section{Benefits}{}%
Why would Ben or Amy submit to this approach? Because it makes their
life a good deal easier. For Ben, it's easy to distribute is files in
such a way that package managers and indexers can easily find them
(The Semantic Web), and he doesn't have to do anything out of the
ordinary, since he would have had to write a similar file in his
implementation's module language anyways. Now, he can write the module
forms once in the Packager format, and the other module languages can
be generated from them automatically. 
 
Amy gets the benefit of Ben's choice, because now it's easier for her
to work with the library, and easy for her to integrate it into her
system. With the packager files, she can use automatic tools to do her
management work for her, instead of having to manage it all herself.

\section{How to use this guide}{}%
This guide is a sort of kickstart for the user who justs wants to see
how the Descot Packager works. It's not a complete documentation of
the system, and it doesn't describe all of the low-level details. In
fact, we purposefully leave some of these details out in favor of
giving you a more idealized picture of the world. We'll get into some
of the details in the latter sections, but if you want to get into
those details, the packager is written as a Literate Program
(ChezWEB), and you can get access to the Woven code, which is
essentially the technical documentation and source code in one. If you
really want to see how things are done, I encourage you to investigate
these documents, which will tell you all you ever wanted to know about
the systme.

If you just want to get the idea of how to use the system, and figure
you can learn as you go, you should start with the Tutorial, which
will guide you through the basics. The other sections can then be
investigated or used as a reference whenever you need to use them.
 
\chapter{Tutorial}{}%

It would probably help a little if there was a small example to go by,
right? This chapter will show you the light and easy introduction to
Descot, so you won't get too scared off of the packager. The basics
steps to using the packager are A) have your code written; B) Write
the package files; C) Generate SRDF for uploading to a Descot server;
and, D) Generate implementation modules for yourself. 

\section{Our Example Library}{}%
For our sample library, we're going to take a very simple one, so that
you can see the basics of Descot without having to get into too many
of the details right away. Let's say we have a library {\tt
(my-even/odd)} that exports {\tt even?} and {\tt odd?}, which do the
classic things. Let's say that we've written this in a simple file
called {\tt even\_odd.scm}. That file would contain, perhaps the
following:
 
\medskip\verbatim
(define (even? x)
  (assert (and (integer? x) (not (negative? x))))
  (or (zero? x) (odd? (- x 1))))

(define (odd? x)
  (assert (and (integer? x) (not (negative? x))))
  (if (zero? x) #f (even? (- x 1))))
|endverbatim
\medskip

\section{Creating the Package files}{}%
In the above code, we want the library for this code to export both
procedures. We also have some dependencies. This is R6RS code, so
we'll assume a dependency on the {\tt (rnrs)} library. We'll also
throw some other meta information in there to make it easy to find.
Here's how an example package files might look:
 
\medskip\verbatim
(node (my-even/odd) : Library
  (name "Even and Odd")
  (description
    "Some generic implementations in R6RS of EVEN? and ODD?")
  (copyright-year 2010)
  (copyright-owner arcfide)
  (license public-domain)
  (import (rnrs))
  (export even? odd?)
  (location 
    ,(blank-node Single-file 
       (url "http://me.com/even_odd.ss"))))
|endverbatim
\medskip

\noindent
While this is a pretty simple example, there are a few special points
to mention while we're here. Notice the use of {\tt unquote} here. In
this case, a location property usually gets resolved into a URL.
However, we wanted to have a little more control over this, so I used
{\tt unquote} to tell the property to stop resolving and just evaluate
the Scheme code as normal, without any resolving. this means that {\tt
blank-node} is treated as a syntax just like the rest of the forms,
like {\tt node}, for example. This way, I can create a blank node
here, that is a Single-file, and that points to where the file can be
grabbed, without having to place it outside and give it a name. This
is the main use of {\tt blank-node}, as it has almost no meaning
outside of an internal use like this.
 
If you don't like {\tt unquote}, then it is possible to use another
form instead, by using the {\tt escaper} syntax parameter to do it.
That is beyond this tutorial though, so read the rest of the Guide if
you want to learn how to do that. 
 
Let's see how the others work. The {\tt name} property here is a
string property, so it expects that we give it a string as its object.
The {\tt import} and {\tt export} properties are called node
properties, which means that they take their objects and resolve them
into URIs. 

``But wait,'' I hear you say! How does the resolving work? Well,
that's where the good stuff comes in. Each class that we use is
associated with a root parameter. We can control how the node names
are resolved into full URIs with these root parameters. In the case of
{\tt import} and {\tt export}, the properties use the {\tt
Library-root} parameter and the {\tt Binding-root} parameter,
respectively, which are the root parameters for the {\tt Library} and
{\tt Binding} classes. 
 
A node object can be either a string, in which case that whole string
is taken to be an already resolved, complete URI, or it can be a
symbol or list of symbols. When it's a symbol or list of symbols, then
the node is resolved by turning the symbol or list of symbols into a
URI suffix, and prefixing the specific root onto the suffix. So, if
you have, say, a node like {\tt (my library)}, then the suffix might
becomes {\tt "my/library"} and the prefix, whatever it may be, would be
attached to this. 
 
Customarily, you should set up all of your root parameters ahead of
time in a separate prelude file, so that you could reuse it for
multiple libraries. Here's an example that one might use:
 
\medskip\verbatim
(default-root "http://rdf.blah.com/")
(Library-root "http://rdf.blah.com/libs/")
(Person-root "http://rdf.blah.com/people/")
(Binding-root "http://rdf.blah.com/bindings/")
(License-root "http://rdf.blah.com/licenses/")
|endverbatim
\medskip

\noindent
And that's pretty much all there is to the Descot Packager!

\section{Generating SRDF}{}%
After we have created this Package file, the next step is to generate
an SRDF file from it so that we can send this to a Descot Server.
 
\medskip\verbatim
$ descot-packager srdf prelude.dsct even_odd.dsct > even_odd.srdf
|endverbatim
\medskip

\noindent
Well that was easy, now, wasn't it? 
 
\section{Generating Implementation Modules}{}%
If we want to use this package file as a module system in our Scheme,
we should probably grab the right converter module for our system,
say, in this case, R6RS. Once we install it into the right module
path, we can generate our module declarations:
 
\medskip\verbatim
$ env DSCTPKGRMODPATH=<path_to_module> \
    descot-packager r6rs prelude.dsct even_odd.dsct
|endverbatim
\medskip

\chapter{Packager Programs}{}%
Currently there is one program for the Descot packager: {\tt
descot-packager}. This section details the program(s) and will give
you an idea of the general user interface to the packager programs. 
 
\section{{\tt descot-packager}}{}%
The packager program
defined here processes packager input files and generates output to standard
output of the given type specified by <output-type>. At least the 
output-type SRDF is supported. Additional modules supporting different 
output types can be loaded dynamically based on what is available.

\medskip\verbatim
packager: <output-type> <input-file> ...
|endverbatim
\medskip

\noindent
The {\tt <output-type>} should be either SRDF of the name of a module
in the module path directory, which can be changed by altering the
DSCTPKGRMODPATH environment variable. Each input file should be a
valid Descot Packager file as defined by the syntax in the language
module of the Descot source or the following format section of this
document. 
The program will search for the {\tt <output-type>} if it is not SRDF
and will dynamically load the module if it is found. It will exit with
an error if the module does not exist. 
Input files are processed from left to right, and the results of
processing are printed to the standard output. 
 
\chapter{Packager File Format}{}%
Descot Packager is a utility to assist in the creation and use of Descot
stores. It is intended to be used as an easier packaging format for 
programmers who write libraries. After writing the code for the library,
the programmer can write a Descot package description, which can be 
then turned into an SRDF file for submission to Descot repositories, 
or, it can be used to generate implementation specific Module declarations.
It contains the Descot Package Language, Class Definitions, Constructing 
Subject URIs, Properties, Grounding/resolving relative references, Typed 
Properties, Descot Property Definitions and Descot Node URIs.

\section{Syntax}{}%
For the Descot Package Language, it is an encoding of a specific RDF
based language in prefix notation. It allows for easier and more natural 
construction of Descot stores without having to do all of the heavy syntax 
work that SRDF, XML, or Turtle would require.Essentially, to build a Descot 
Package file is to define a graph by specifying the association or edges 
associated between a given subject and a series of objects. That is, a Descot 
package file consists of global definitions and node forms. These node forms 
associate a particular subject of a given type with zero or more properties. 
So, a package file consists of a series of the following forms:

\medskip\verbatim
<definition>
(node <subject> : <class> <property> ...)
|endverbatim
\medskip

\noindent
The combination of all of these associations forms a graph of the 
various nodes and edges given in the file(s). The output of these 
files is SRDF code, and these forms can be thought of as a syntactic 
wrapping around SRDF. That is, they have rather direct mappings to 
SRDF equivalents, provided that allow for the specfic syntactic 
limitations imposed by each. They are not as general as SRDF forms.

However, since these forms simply represent a special Scheme evaluation 
environment that can be used to build an SRDF expression list, it is also 
reasonable for allowing that SRDF can be listed in any place where you 
can place a raw Scheme value and have it interpreted. This would be at the 
top level and anywhere that is escaped by an escape form. The quoted SRDF datum
may then be given directly, instead of relying on the specfic syntactic forms 
given here. This allows using the syntactic forms whenever possible, but
allows breaking this form and use SRDF when necessary.

For Library Definition, there is a library that form the execution environment
of our Descot packages. It contains classes like Library, Library-root and 
Library-default-properties, Binding, Binding-root, Binding-default-properties 
and so on, which could be expressed in the form of Class, Class-root, Class-
default-properties. In addition, there will be lots of other properties being 
included, like Implementation-default-properties, default-root, parameterize,
escaper and so on.

For Class Definitions, whenver create a new node in a Descot graph, the node is 
almost certainly associate with some particular class. Each class 
provides information to nodes about how they should behave and what valid 
properties can be used. Within the Packager language, these nodes are used 
to provide roots and property defaults for resolving node names and 
inserting properties for each class of nodes.

At the moment, classes are used only to resolve the subjects and fill 
in default properties. It could also be used to do error checking on 
the properties, but this is not implemented yet.

For each class, subject root and default properties parameters are provided, 
which establish the defaults for the class, as well as the class type, which 
is used to automatically insert the correct type annotation on the class 
subject. Subjects, if relative, will use the subject root parameter to fully 
ground their locations, and the default properties will be added to this class, 
and can be defined globally. Plus, a default-root property that can be used as 
a fallback since even though a root parameter for every class is defined, it's 
likely that some of these default root parameters will not be defined. The 
subject root parameter contain a string that is the default base root for any 
relatively defined subject with the given class.

The default properties parameter is bound in the prelude file to 
any properties true that hold for all subjects of a particular class.
There are procedures enforce the naming conventions, with these procedures, 
each of the classes could then be trivially defined. However, all of the
classes have a similar naming convention, so to enforce this, a bulk class 
definition syntax should be used. Basically, each class should be a 
capitalized term, which forms the prefix for the two parameters, namely, 
root and default-properties. Each should be prfixed with the {\tt Class-}.

Finally, it is assumed that each class definition has a node name defined 
by {\tt (dscts "Class")} which is the type of the class in RDF.
The following is a nice looking set of classes definition.

For Constructing Subject URIs, when a {\tt node} form is used, the subject
may be in either a string, symbol,or list form. In the latter two forms, 
the object or subject is considered to be a relative reference. This
means that it must be converted to a URI path and then appended to the 
base URI. A string reference is expected to be an absolute reference.

For Creating new Descot nodes,with a given set of properties, we use node.
Creating a new node consists of providing a subject URI, its type or class, 
and then the set of properties and objects associated with that subject node.
When this form is expanded, it expands into an SRDF literal whose subject 
is resolved using the root from the <class> and has at least the default 
properties given by the <class> default properties parameter.

For Descot Node URIs, all Descot nodes are prefixed with the same URI 
prefix, and a helper is used to make sure that the prefix is thrown in 
at the appropriate points.

\section{Properties}{}%
For Properties,they accept as their arguments forms which are meant to 
indicate the type of object associated with a given property.The property 
will be some URI, and the object will be some proper object list. In order 
to define a property, some syntax are defined, which accepts some 
syntax for the arguments and then performs some transformation on it to 
build the object. Additionally, we provide the property's URI.The 
property-syntax-rules macro makes sure to check the input clause forms,
then creates a syntax-rules form.

For Grounding/resolving relative references, properties will usually 
have to ground their relative object references in some way or another
based on some root URI.

For Typed Properties, there are a number of common idioms for properties 
that show up: {\tt string-property} expects a single string argument as 
the object, {\tt date-property} expects a single date string in a standard 
RDF dateTime format, {\tt year-property} is used for handling cases that 
expect the year, {\tt node-proprety} expects a single node,a single node is 
normally resolved using the resolve procedure defined above, but it is also
possible to use and escape keyword to evaluate arbitrary code and have the 
result inserted as an object, {\tt node-list-property} defines a property 
that expects a list of nodes as its objects,it ensures that the unquote 
case is handled just as in the single node property definitions, the 
{\tt node-properties} syntax mentioned above needs to create a list of 
the object URIs, {\tt string-list-proprety} expects one or more 
strings as objects, {\tt type} is a unary procedure that take some class type 
and returns a SRDF predicate of the form assuming that rdfs is the 
standard RDF Syntax prefix, {\tt xsd} is a node constructor that creates XSD 
nodes using the appropriate prefix.


\section{Classes}{}%
The Descot metalanguage defines the following classes:

\unorderedlist
\li Library
Library class represents libraries, and most of the properties
stem from Library nodes. Library nodes are also the
main root node for most retrievals.
\li Binding
It contains information about a procedure or macro that is 
exported or imported from a library.
\li License
It contains information about a License type, such as ISC, 
BSD, GPL, or a proprietary license of some sort. They may 
point somewhere else as the main reference, and have only a 
short description of the actual license in the graph, or 
they may contain the entire text of the license as the description.
\li Person/Implementation
Person and Implementation nodes follow a similar pattern,
describing people and implementations, respectively.
\li Retrieval-method
A node for describing a method to give files, maybe a single file 
or archive and so on.
\li Archive
It contains file archive download information.
\li Single-file
It is similar to Archive nodes, but they point to single Scheme 
files instead of archives.
\li SCM
Is a general class for ��Source Control�� based libraries.
That is, SCM is a sub-class of Retrieval-method like Archive
and Single-file are, but it describes a retrieval via some
source control module, like CVS. 
\li CVS
CVS is the sub-class of the SCM class that describes CVS 
server modules particularly.

\endunorderedlist

\section{Escapes}{}%
When writing node properties, each node is essentially a quoted form 
that is not evaluated,but analyzed for its shape and converted and 
used as a datum. Sometimes, you may want to pass a datum directly through, 
and you may want to have that datum obtained as the result of evaluating
the form instead of treating it like a datum. Escapes provide a way for 
you to do this. By default, unquote is used as the escaper, but this 
section defines the forms used to control what escape keyword is used.
Two forms are defined in this section: escaper and with-escaper. 
The escaper form works like a syntactic paramater. The with-escaper 
form allows you to rebind the escaper parameter for a specific
duration.
When using the escaper syntax paramter directly, evaluating the 
form (escaper) will result in the current escaper identifier, whereas 
doing something like (escaper jump) will set jump to be the new escape
keyword. This keyword controls how node properties handle their nodes. 
So, if you have a node property prop that you want to pass something 
directly, by evaluating it instead of treating it like a node form,
you might do something like this:

\medskip\verbatim
(escaper lift)
(prop (lift (list 'a 'b 'c)))
|endverbatim
\medskip

\noindent
The with-escaper form sets and resets the escaper syntax parameter
for the duration of the body code, which it splices into the existing 
code, you can wrap many forms with the with-escaper form, so it is 
possible to parameterize the escape form over a number of node definitions.

\chapter{RDF Primer}{}%
Underneath it all Descot is a metalanguage written in RDF for source
code. We try to hide these details as much as possible because, let's
face it, no one likes to write RDF. However, RDF gives us some
significant advantages when it comes to extensibility and flexibility.
It also simplifies our model of working with things, though it
requires adding a few layers of indirection to get something that is
also simple to use in the front-end. 
 
This chapter is designed to give you the basics that you need to work
with RDF in Descot. There are some basic concepts that you'll want to
know, and some basic techniques with which you should be familiar. If
you learn those, you shouldn't have to bother with the nitty gritty of
RDF until you want to, and hopefully, you'll never be forced to do so.

\section{What is RDF?}{}%
RDF itself stands for ``Resource Description Framework.'' RDF isn't
an encoding, a file format, or anything concrete at all. In reality,
it's an abstract idea of how to relate concepts, and a vocabulary for
relating these ideas. The basic idea behind RDF is to represent
information about resources as a directed graph. This graph has a set
of nodes, and some of these nodes are connected to other nodes by
edges, which are called properties or verbs in RDF lingo. The nodes
have a direction, so that the edge goes from one node to another. The
originating node is called a subject, while the target node is called
the object. Thus, you have a bunch of nodes and arrows pointing to
various things, and each of them has a name. 
 
When you want to identify a specific node or edge, you do so by its
name. Names in Descot are URIs. Yes, you heard that right. RDF uses
URIs to uniquely identify nodes in a global fashion. Not all nodes
must have URIs, but the only ones that don't are called blank nodes,
which must be connected to other nodes, and you can not really access
blank nodes except by going through some other URI-named node. 
 
However, directed graphs, no matter how nice, don't really do us much
good if we can't write them down. That's why various encodings exist
for writing down RDF graphs into files or representing them in some
serialized fashion. We'll get to these more in a moment.
 
Before we get to writing an RDF graph, we should ask ourselves what a
graph means. To interpret an RDF graph, you have to see how the Schema
defines the nodes that are used. There isn't much explicitly to
indicate how a node should be used, but there's usually some intention
behind it. Nodes themselves may actually represent objects in the real
world, such as a book, or in our case, source code. You can then have
arrows going out of this node, which are properties relating that node
to some other node in some fashion. The names of these properties will
usually indicate how these two nodes relate. For example, an arrow or
property might be a name property, that says that the subject has the
name of the object, which might be a string literal. So, in this way,
it's almost like an association list, where the keys are properties
belonging to something. In Scheme, we might look to property lists as
an example of how this might work. We would use symbols instead of
URIs, and the properties would be the keys in the property list, and
the subject would be the symbol to which the property list is
attached. 
 
At any rate, you can think of an RDF graph as being a bunch of
sentences with a subject, verb or property, and object, and each verb
is saying something about the subject and the object and how they
relate. Let's see how we use this in Descot proper.

\section{How Descot uses RDF}{}%
In Descot, we have a meta language that is a vocabulary of Classes and
properties that extend the basic RDF vocabulary. Using this
vocabulary, we can build up information about source code. We also
have SRDF, which is an S-expression based encoding of RDF. We have the
packager format, which is a simpler language that uses the Descot
vocabulary to describe source code resources. The package format is
translated into the more general SRDF form. Indeed, you can mix and
match SRDF with the package format. 
 
In Descot, we have properties for things like the name of the node,
which might be a node representing some particular library that you
have written. It might also be a node representing a Scheme
implementation. We use Classes to distinguish the types of each of
these nodes. 

\section{How Packages map to RDF Graphs}{}%
Each package or set of package files in the packager format encodes
some RDF graph. The properties are edges in the graph, and the nodes
are created with the {\tt node} form. The above chapters discuss how
this all is used together, but the main point is that when you write a
package format, you should remember that you are writing a graph of
information about your source code, and that is used by the packager
and other Descot-compliant tools to deduce information from the
packages to do things like install them, or grab their dependencies
automatically, or create Scheme implementation specific modules
derived from them.

\section{Dealing with namespaces}{}%
The biggest headache in RDF is the URIs. In Descot, we also assume
that each library, each Binding, each person, and so forth, is going
to receive some unique RDF URI that uniquely identifies it in the
world. You can cheat if you know that your world is going to be very
small, but otherwise, you should spend a little time, at least,
thinking about how you are going to name all of your nodes. Nodes are
created for just about everything. Literal nodes don't need URIs, but
the other nodes do. It generally makes sense to have all of the URIs
relating to Libraries or belonging to the class of Libraries be
extensions of one URI, and so forth for each Class of nodes.
Generally, you'll want to create a hierarchy as you have seen in the
Tutorial, with roots defining folders to store each of your nodes, as
it were. These ``folders'' are really just prefixes for all of the
URIs, but these prefixes help to uniquely identify your nodes when
they get mixed in with others. Generally, Descot requires that these
nodes be referencable according to the Descot protocol, so that a
Descot system can access the information that might have been missing
form the graph. If you know that you won't need that, such as if you
are packaging things up purely for module generation purposes, and do
not ever plan on submitting the meta information to a Descot server,
you might be able to get away without prefixes, but I don't recommend
it. 
 
So, the most time you'll spend thinking about RDF in Descot will be
when you have to come up with the prefix roots for your nodes. I like
to have something like {\tt <service>://<domain>/<default>/<class>/}
as my URI prefixes, because it forms a nice tree. 
 
Don't forget that you might also want to look to other people who may
have already put up the meta data they need, such as for Scheme
implementations, or shared library information. One of the benefits
here is that you don't have to do the work all on your own. You can
point to another library somewhere else and let other people maintain
that stuff. Descot programs will know how to handle this cross-domain
and cross-site linking.
 
I hope that makes everything clear!



\bye
