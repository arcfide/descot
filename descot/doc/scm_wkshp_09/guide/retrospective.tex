\documentclass{sigplanconf}

\title{[your title goes here]}

\authorinfo{[your name here]}{[your affiliation here]}{[your email here]}

% Leave this alone.

\conferenceinfo{2008 Workshop on Scheme and Functional Programming}{}

% Leave this alone.  The program chair will adjust the page number
% and cause page numbers to be printed at the bottom of each page.

\setcounter{page}{1}

\begin{document}

\maketitle

\begin{abstract}
Converting decimal scientific notation into binary
floating point is nontrivial, but this conversion can
be performed with the best possible accuracy without
sacrificing efficiency.
\end{abstract}

\section{Introduction}

Having learned to count on their fingers, humans like
to express real numbers in decimal scientific notation.
Most computers are designed to calculate using numbers
that are expressed in IEEE-standard binary floating-point
notation.

Although every binary floating-point number can be expressed
in decimal scientific notation by using enough digits, most
numbers that are expressible in decimal scientific notation
cannot be expressed in binary floating-point.  For example,
0.1 is not expressible in binary floating-point.  The value
of the closest IEEE double precision floating-point
approximation to 0.1 is
\[
  .1000000000000000055511151231257827021181583404541015625
\]
A decimal-to-binary conversion routine that always delivers
the closest floating-point approximation to its input, and
breaks ties according to the current rounding mode (typically
round-to-even), is said to perform \emph{correct rounding}.

My PLDI paper gave the first description of an efficient
algorithm for correctly rounded decimal-to-binary conversions
\cite{Clinger:1990:HRF}.
Section 10 of that paper describes its motivation and its
relationship to the paper by Steele and White in that PLDI
and this collection \cite{Steele:1990:HPF}.

IEEE-conforming decimal-to-binary conversions have been allowed
to lose almost twice as much accuracy as would be lost by correct
rounding \cite{IEEE-754}.  When my PLDI paper was published, almost
all implementations did lose this much accuracy at times.
This source of inaccuracy could conceivably affect the result
of a numerically unstable computation.  More importantly,
this loss of accuracy has made decimal scientific notation less
attractive for exchanging numerical data between different
systems \cite{priest-appendixD}:
\begin{quote}
Unfortunately, the IEEE standard does not guarantee that the
same program will deliver identical results on all conforming
systems.  Most programs will actually produce different results
on different systems for a variety of reasons.  For one, most
programs involve the conversion of numbers between decimal and
binary formats, and the IEEE standard doesn't completely specify
the accuracy with which such conversions must be performed.
\end{quote}

In consequence of the research described below,
most computer systems now perform correct rounding, and
several language standards now require correct rounding of
decimal-to-binary conversions.  The committee that is revising
the IEEE standard for binary floating point arithmetic is
considering a proposal to require correct rounding.


\section{Implementations}

Within months of PLDI '90, David Gay improved upon my Algorithm
Bellerophon by replacing the extended precision floating point
calculation by a standard-precision floating point calculation
combined with a high-precision integer calculation \cite{Gay:1990}.
Gay also noted an easy case that I had missed, simplified the
algorithm by using a uniform error bound for all hard cases,
and reduced the table sizes.

Gay also improved upon Steele and White's Dragon algorithm.  He
implemented both improved algorithms in C, and made his code
available for anyone to use \cite{Gay:netlib}.  Gay's code was
slower than previous conversion routines on hard cases (for
which the previous routines were often less accurate), but was
so much faster on typical cases that Gay's code was faster
overall.

Gay's code was also more robust and more accurate.  These
advantages were demonstrated by David Hough's \emph{Testbase}
program, and were documented in a manuscript written by Vern
Paxson under the direction of William Kahan
\cite{Hough:testbase,Paxson:1991}.
Most major workstation vendors soon incorporated Gay's code
into their standard libraries.

The implementation of correctly rounded decimal-to-binary
conversion for the IBM S/390 apparently began with my
algorithm instead of Gay's code \cite{Abbott:1999:ASS}.

Meanwhile I had implemented correctly rounded conversions in
Scheme for MacScheme, which ran on the Apple Macintosh, and
made my code available to other implementors of Scheme and
Common Lisp \cite{macscheme}.  Robert Burger and Kent Dybvig
implemented correct rounding for Chez Scheme, and made their
code available to other implementors \cite{Burger:1996:PFN}.
Most major implementations of Scheme now provide correctly
rounded conversions.


\section{Standards}

The Scheme standards cite the PLDI '90 papers, and require
numeric conversion routines to preserve the value of a
number across a \emph{round-trip} of output followed by input,
but do not actually require correct rounding
\cite{IEEE-1178,r5rs}.
Scheme also requires the output routine, for each individual
number, to generate the minimum number of digits that allows
this round-trip requirement to be satisfied.  This makes
Scheme's round-trip requirement more stringent than the
round-trip requirement of the IEEE floating-point standard,
where the number of digits that are needed to avoid loss of
accuracy during a round-trip is independent of the number.
Hence implementations of Scheme cannot just rely on
IEEE-conforming conversion routines.  The best way to
satisfy Scheme's i/o requirements is to provide correctly
rounded conversions.

It appears that Java was the first programming language to
require correctly rounded decimal-to-binary conversions
\cite{java:api}.  The specification of
{\tt java.lang.Double.valueOf(String)}, for example, says
that a syntactically correct input string
\begin{quote}
is regarded as representing an exact decimal value in the
usual ``computerized scientific notation''; this exact
decimal value is then conceptually converted to an
``infinitely precise'' binary value that is then rounded
to type double by the usual round-to-nearest rule of
IEEE 754 floating-point arithmetic.
\end{quote}

XML Schema, which was approved as a W3C Recommendation on
2 May 2001, requires correct rounding of decimal-to-binary
conversions, citing both my paper and Gay's
\cite{Clinger:1990:HRF,Gay:1990,xml-schema}.

The committee that is revising the IEEE 754 standard for
binary floating-point arithmetic has already voted to
encourage correct rounding of all binary-decimal conversions.
A proposal that would actually require correct rounding has
been written and will soon be considered by the committee
\cite{IEEE-754R}.

 
\section{Correction}

In section 9 of my PLDI paper, I reported that ``some compilers
do not implement IEEE arithmetic correctly.''  This was an
overstatement, as the IEEE 754 standard concerns itself
primarily with the low-level operations as they would be
implemented in hardware or in library routines, and does
not specify many of the language-level and compiler-level
details that determine the behavior of floating-point
arithmetic as seen by most programmers and users.  In
particular, ``the IEEE standard requires that each result
be rounded correctly to the precision of the destination
into which it will be placed, but the standard does not
require that the precision of that destination be determined
by a user's program'' \cite{priest-appendixD}.


\section{Acknowledgements}

I have relied on BibTeX entries that were collected or
written by Guy Steele and Nelson Beebe for the committee
that is revising the IEEE 754 standard
\cite{IEEE-754R,Knuth:1990:SPW}.


\bibliographystyle{plain}
\bibliography{refs2,refs}

\onecolumn

\end{document}
