SRDF is an s-expression based format for describing RDF Graphs.  It is
meant to be mostly equivalent in its form to Turtle. Since the language
is S-expression based, it is easier for Scheme and Lisp parsers to
parse it. Parsers for other languages can also be written very
easily. This makes it particularly nice for use in automated systems
or in areas where S-expressions are the natural representation format.
SRDF is designed to work for most Scheme's {\tt read} procedure. 

SRDF documents are composed of a series of RDF triples and possibly,
prefix definitions. Prefixes take the form
{\tt (= name "uri")}, and associate a given Scheme
symbol with a URI string. Otherwise, the form is an RDF triple or a set
of triples.

Normal triples are just a list of three elements, each a URI. However,
it is possible to have a subject associated with more predicate
and objects by replacing the list that would hold the single
predicate and object with a list of such predicates and objects.
Likewise, one can specify more objects to be associated with a
given subject and predicate by doing the same thing with the
object list, and replacing the {\tt cdr} that would normally hold the
object with a list of such objects.

If the {\tt cadr} of a predicate pair contains a list of objects, this
represents a collection of objects, and is created in the same
way that a turtle collection syntax is created: by associating
a series of blank nodes with the right predicates with each of
the objects listed.

It is important to note the difference between an object that
is a collection, and a list of objects that are each associated
with the subject and predicate. The following is an instance of
the former:

\medskip\verbatim
("subject-uri" "predicate-uri"
  ("object1" "object2" ...))
|endverbatim
\medskip

\noindent Whereas the following is an instance of the latter:

\medskip\verbatim
("subject-uri" "predicate-uri"
  "object1"
  "object2"
  "object3")
|endverbatim
\medskip

Normal RDF triples take the form:

\medskip\verbatim
("subject-uri" "predicate-uri" "object-uri")
|endverbatim
\medskip

A blank node may be inlined into the graph by using a `*' as the
beginning symbol in an object context like so:

\medskip\verbatim
("subject" "pred" (* "pred" "object"))
|endverbatim
\medskip

Of course, blank nodes may have anything that is a valid predicate
{\tt cdr} as its {\tt cdr} so the following is also valid:

\medskip\verbatim
("subject" "pred" 
 (* ("pred1" "object1") ("pred2" "object2")))
|endverbatim
\medskip

URIs may be described by their full path names as strings,
as prefix combined paths, or as blank node paths. The following
are all valid URIs:

\medskip\verbatim
"http://some.domain/path/to#blah"
"blah"
(: prefix "blah")
(_ "uniqueid")
|endverbatim
\medskip

We use `:' for prefixes and `\_' for blank nodes.

In addition to URIs, we permit literals as valid {\tt car}s for objects.
Literals can be strings, numbers, booleans, or may be strings
with either languages or types associated with them. The following
are examples of languages and types, respectively:

\medskip\verbatim
($ "Language unspecified.")
(& "English Sentence lies here." en)
(^ "2008/01/03 14:00" (: xsd "date"))
|endverbatim
\medskip

The following is a fairly formal BNF grammar with the exception of
tokens such as strings, numbers, and booleans being undefined and
presumed to be defined lexical values. Additionally, we define
S-expression in terms of atoms and pairs, so the BNF grammar is
also defined in the ``longhand'' notation for pairs and lists.
This means that while the BNF Grammar states something
like {\tt ("subj" . ("pred" . ("obj" . ())))} as the
valid simplistic RDF triple, it is also legal in practice to use
the shorthand version of this: {\tt ("subj" "pred" "obj")}

\begingroup\narrower
\begingrammar
%
<rdf sexp>:     <rdf triple> | <rdf triple> <rdf sexp>.\par

<rdf triple>:   "(" <uri> "." <rdf subject tail> ")" ; | "(" "=" name string ")".

<rdf subject tail: <rdf predicate> ; | <rdf predicate list> .

<rdf pred list>: "()" ; | "(" <rdf predicate> "." <rdf pred list> ")".

<rdf predicate>: "(" <uri> "." <rdf object list> ")".

<rdf object list>: "()" ; | "(" <rdf object> "." <rdf object list> ")".

<rdf object>: <uri> | <literal>
        ; | <rdf object list> | <blank node list> .

<blank node list>: "(" "*" "." <rdf subject tail> ")".

<uri>: uri ; | "(" ":" "." <uri list> ")" ; | "(" "\_" name ")" .

<uri list>: "()" ; | "(" <uri name> "." <uri list> ")".

<uri name>: uri | name.

<literal>: number | boolean ; | "(" "\$" string ")" ; |
        "(" "\&" string name ")" ; | "(" "\^{}" string <uri> ")".

\endgrammar
\endgroup
\endgroup
\medskip
